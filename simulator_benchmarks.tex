

\documentclass{article}
\usepackage[utf8]{inputenc}
\usepackage{courier}

% allow appendix sections after a certain delimiter
\usepackage[title]{appendix}

\usepackage[english]{babel}
\usepackage{hyperref}	
\usepackage{graphicx}% Include figure files
\usepackage{dcolumn}% Align table columns on decimal point
\usepackage{bm}% bold math
\usepackage{hyperref}% add hypertext capabilities
%\usepackage[mathlines]{lineno}% Enable numbering of text and display math
%\linenumbers\relax % Commence numbering lines
\usepackage{braket}
\usepackage{amsmath,amssymb,latexsym}

%modulo command
\newcommand{\Mod}[1]{\ (\mathrm{mod}\ #1)}


\usepackage{bbold}
\usepackage[usenames, dvipsnames]{color}

% integer command is \Z
\usepackage{amsfonts}
\newcommand{\Z}{\mathbb{Z}}


\usepackage{comment}
\usepackage{natbib}
\usepackage{mathrsfs}
\usepackage{soul}
%textboxes
\usepackage{mdframed}
\newcommand{\spliteq}[1]{\begin{equation}
\begin{split}
#1
\end{split}
\end{equation}
}
\newcommand{\code}[1]{\texttt{#1}}

\newcommand{\dd}[2]{\frac{d#1}{d#2}}

\usepackage{qcircuit}


\title{Performance benchmarks for common circuit simulators}
\author{Evan}
%\date{October 2018}





\begin{document}

\maketitle

\begin{abstract}
	abstract here 
\end{abstract}


\newpage
\tableofcontents
\newpage



\section{Methods}
\section{Simulator methods}

\subsection{General unitary matrix multiplication}
\label{sec:fallback3}
A complete wavefunction simulator (cite Biamonte for what that means; I'll be reffing tensor networks anyways) generally applies the action of a gate as an einsum to the qubit axes acted on by the gate.

Worked example: Computing $U_2 U_1 \vec{\psi}$
For example, let $U_1$ be a unitary over a subset of $k_1$ and $U_2$ be qubits on a state defined over $k_2$, and the target state $\ket{\psi}$ be defined over $n$ qubits. Then we have:

\begin{align}
	 \vec{\psi} &\in \mathbb{C}^{2^n} \\
	 U_1 &\in \mathbb{C}^{2^{k_1} \times 2^{k_1}} \\
	 U_2 &\in \mathbb{C}^{2^{k_2} \times 2^{k_2}} \\
\end{align}


The following list details approaches to computing $U_2 U_1 \ket{\psi}$ in order of worst to best:
\begin{enumerate}
	\item \textbf{kronecker product + matmul}: This requires resizing the matrices to $n$ dimensions then performing matrix multiplication
	\begin{enumerate}
		\item kronecker product $U_1 \rightarrow U_1' \in \mathbb{C^{2^n \times 2^n}}$ using $I_2$ ($\mathbb{O}(2^k_1 2^k_1 2^{n-k_1} 2^{n-k_1}) = \mathbb{O}(2^{2n})$)
		\item kronecker product $U_2 \rightarrow U_2' \in \mathbb{C^{2^n \times 2^n}}$ using $I_2$ ($\mathbb{O}(2^k_2 2^k_2 2^{n-k_2} 2^{n-k_2}) = \mathbb{O}(2^{2n})$)
		\item matmul $U_2' U_1' \rightarrow U_f \in \mathbb{C^{2^n \times 2^n}}$ ($\mathbb{O}(2^{3n})$)\footnote{Assuming ``schoolbook'' matrix multiplication algorithm. For square matrix multiplication this can be improved to $\mathbb{O}(2^{2.807n})$ or even $\mathbb{O}(2^{2.37n})$ with ML but that's not the point of this worst-case example}
		\item matmul $ U_f \vec{\psi}$ ($\mathbb{O}(2^{2n})$)
	\end{enumerate}
	\item \textbf{kronecker product + matmul, associativity}: This swaps (c) and (d), taking advantage of the fact that $U \vec{\psi}$ has complexity $\mathbb{O}(2^{2n})$ for $U \in \mathbb{C}^{2^n \times 2^n}$, meaning both products cost $\mathbb{O}(2^{2n + 1})$ instead of $\mathbb{O}(2^{3n} + 2^{2n})$. Conceptually this results from not having to calculate the full intermediate $U_f$.
	\item \textbf{einsum}: An einstein summation allows general tensor transformations using repeated (summation) and permuted (free) indices (See Appendix~\ref{sec:a1} for some common examples). The austere implementation results in a complexity of\cite{https://obilaniu6266h16.wordpress.com/2016/02/04/einstein-summation-in-numpy/}: 
		\begin{equation}
			\mathbb{O}\left( \left(\prod_i^{N_\text{free}} d_i\right) \left(\prod_i^{N_\text{sum}} d_i\right) \right)
		\end{equation}
		where $N_\text{free}$ is the number of free indices (indices of output), $N_\text{sum}$ is the number of unique input indices between input tensors, and $d_i$ is the size of the axis corresponding to the i-th index. \footnote{For example, the einsum string \textbf{ik,kj-$>$ij} multiplies two matrices using free indices \textbf{i}, \textbf{j} and summation index \textbf{k}; if \textbf{i}, \textbf{j}, \textbf{k} index axes of sizes $\ell,m,n$ respectively, the complexity of this einsum is 	$\mathbb{O}( (d_i d_j) (d_k) ) = \mathbb{O}( \ell mn)$, the expected complexity of matrix multiplication.}
		
		Let unitaries and wavefunctions be represented as tensors of the shape:
		\begin{align}
			U &\in \mathbb{C}^{\overbrace{ \scriptstyle 2 \times 2 \times \cdots}^{\text{k times}}} \\
			\vec{\psi} &\in \mathbb{C}^{\overbrace{ \scriptstyle 2 \times \cdots}^\text{n times}}
		\end{align}
		with the restriction that $n \geq 2k$ to ensure compatible the state and matrix are compatible for multiplication. Then the einsum computing $U\vec{\psi}$ is indexed as\footnote{Three worked examples:
			\begin{enumerate}
				\item single-qubit matrix $M$ acting on two-qubit state $\psi$ ($k=1$,$n=2$): $(M\psi)_{jk} = M_{ij} \psi_{ik}$ has one summation index \textbf{i} and two free indices \textbf{j,k}, for $\mathbb{O}(2^{3})$
				\item two-qubit matrix $M$ acting on two-qubit state $\psi$ ($k=2$,$n=2$): $(M\psi)_{\ell m} = M_{jk\ell m} \psi_{jk}$ has two summation indices \textbf{j,k} and two free indices \textbf{$\ell$,m}, for $\mathbb{O}(2^{4})$
				\item two-qubit matrix $M$ acting on three-qubit state $\psi$ ($k=2$,$n=3$): $(M\psi)_{\ell m n} = M_{jk\ell m} \psi_{jk n}$ has two summation indices \textbf{j,k} and three free indices \textbf{$\ell$,m,n}, for $\mathbb{O}(2^{5})$
			\end{enumerate}		}
		$$i_1 \cdots i_{k}, i_1' \cdots i_{k}' i_1 \cdots i_n \rightarrow  i_{k + 1} \cdots i_n i_1' \cdots i_k ' $$ has $k$ summation indices and $(n-2k)$ free indices, resulting in a complexity of $\mathbb{O}(2^{n + k})$ since every tensor axis is dimension two. 
\end{enumerate}

The complexity of method (3) does not saturate the lower bound for computing all of the elements in $\vec{\psi}' = U \vec{\psi}$ elements, which can require as few as $2^{n}$ operations (namely, $I \vec{\psi}$ using sparse multiplication saturates this bound). This motivates even more streamlined unitary action, which will be introduced in the following sections.

\subsection{Nonlinear mappings}

\subsubsection{Example: $X_k$}
\label{sec:clifford_example}
This section begins with an example of applying a permutation to a tensor subspace (taken from the \texttt{apply\_unitary} from the \textsc{cirq} library). Let $X_k$ be the Pauli-X gate acting on the qubit indexed ``k''. Define the operation $\texttt{slice\_for\_bitstring}(T, \{i\text{ for i=1...k}\}, \vec{s})$ to access elements of tensor $T$ for which the subspace corresponding to qubits $\{0, 1, \cdots, k\}$ is represented by the (little-endian) bitstring $\vec{s}$ (and so $|\vec{s}| = |\{i_k\}|$). TODO: example of this....

This slices for a subset of $2^{n-k}$ elements. Then the action of $X_k$ on $\vec{\psi}$ can be computed in two steps using a memory buffer $T'$ sized the same as $T$ (namely, $2^n$):
\begin{itemize}
	\item Compute $S_0 = \texttt{slice\_for\_bitstring}(T, \{k\}, (0))$ and \\$S_1= \texttt{slice\_for\_bitstring}(T, \{k\}, (0))$ which slice $T$ for all elements in which the k-th qubit is a ``0'' or ``1'' respectively
	\item Set $T'[S_1] = T[S_0]$ and $T'[S_0] = T[S_1]$ at a cost of $\mathbb{O}(2^{n-1})$ each, for a total complexity requirement of $\mathbb{O}(2^n)$. In plain English, this sets the amplitude for each basis state in $T'$ to be the same as the amplitude in $T$ where the k-th qubit of that basis state is flipped. 
\end{itemize}


Section \ref{sec:clifford_example} demonstrated how a certain single-qubit gate acting on an n-qubit state can be implemented in $\mathbb{O}(2^n)$, which is at least $2\times$ improvement over the best matrix-style operations (at the expense of exponential memory overhead in the number of qubits, i.e. the uninitialized buffer state).


TODO
\subsubsection{General Clifford operations}
- Clifford circuits are easy to simulate (Gottesman). Computationally these are relatively sparse matrices.

Here we review an algorithmic app

\section{Benchmark methods}
This repo uses the \textsc{pytest-benchmark} fixture to profile python code.

\section{Results}


\begin{appendices}
\section{Appendix 1: Einsum operations}
\label{sec:a1}

Some common einsum-compatible operations: transpose, inner produc, matrix multiplication, trace.
\end{appendices}
\end{document}